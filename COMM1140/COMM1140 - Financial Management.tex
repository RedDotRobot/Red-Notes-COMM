\documentclass{article}

% Imports the catppuccin theme, using the mocha flavor,
% from the directory above. Actual implementation
% wouldn't need the import package unless the theme
% and the document are in different directories.
\usepackage{import}
\usepackage{xcolor}
\usepackage{cancel}
\usepackage{mathtools}
\usepackage{hyperref}
\usepackage{setspace}
\usepackage{float}

\newcommand\Myperm[2][^n]{\prescript{#1\mkern-2.5mu}{}P_{#2}}
\newcommand\Mycomb[2]{\prescript{#1\mkern-0.5mu}{}C_{#2}}


\definecolor{yorhabg}{HTML}{C8C2AA}
\definecolor{yorhafg}{HTML}{4D493E}
\definecolor{yorhagrid}{HTML}{B5AF9C}
\definecolor{mred}{HTML}{D24545}

\usepackage[dvipsnames]{xcolor}
\definecolor{pastel_red}{RGB}{255, 173, 173}
\definecolor{pastel_orange}{RGB}{255, 214, 165}
\definecolor{pastel_yellow}{RGB}{253, 255, 182}
\definecolor{pastel_green}{RGB}{176, 217, 176}
\definecolor{pastel_blue}{RGB}{167, 199, 231}
\definecolor{pastel_purple}{RGB}{222, 218, 244}

\pagecolor{yorhabg}
\color{yorhafg}

\import{preamble.sty}

% Removes padding above title
\usepackage{titling}
\setlength{\droptitle}{-10em}

% Font package
\usepackage[T1]{fontenc}

\usepackage{fouriernc}

\usepackage{sectsty}
\usepackage{graphicx}
\usepackage{amsmath}
\usepackage{amsfonts}
\usepackage{amssymb}
\usepackage[skins]{tcolorbox}

\DeclareMathOperator{\sgn}{sgn}

\usepackage{tikz}
\usepackage{eso-pic}
\usetikzlibrary{calc,shadows.blur}
\usetikzlibrary{3d}

\AddToShipoutPictureBG{%
\begin{tikzpicture}[remember picture, overlay,
                    help lines/.append style={line width=0.05pt, color=yorhagrid}]
  \draw[help lines] (current page.south west) grid[step=5pt]
                    (current page.north east);
\end{tikzpicture}%
}

% Margins
\topmargin=0in
\evensidemargin=0in
\oddsidemargin=0in
\textwidth=6.5in
\textheight=9.0in
\headsep=0.25in

\AtBeginEnvironment{tcolorbox}{\small}

\newtcolorbox{question}{%
    enhanced,
    colback=yorhabg,
    colframe=yorhafg,
    coltext=yorhafg,
    coltitle=yorhabg,
    title=\textbf{Question.},
    arc=0pt,
    outer arc=0pt,
    drop shadow southeast,
    sharp corners
}

\newcommand\bb[1]{\textcolor{yorhafg}{\textbf{#1}}}

\title{\textbf{COMM1140: Financial Management}}
\author{L. Cheung}
\date{\today}

\begin{document}
\maketitle
\tableofcontents

\newpage

\section{Assessment Structure}
	\begin{itemize}
		\item \textbf{Assessment 1: Tutorial Participation (15\%)} - Class participation (assessed in tutorials from topic 2)
			\begin{itemize}
				\item Homework Submission (5 marks)
				\item Quality and Impact of Contributions (5 marks)
				\item Frequency of Participation (5 marks)
			\end{itemize}
		\item \textbf{Assessment 2: Group Assignment (25\%)} - Assignment based on the content in topics 1-5 (\textbf{Video Submission in Week 9})
			\begin{itemize}
				\item Released to students \textbf{Monday of Week 5}
				\item Groups of 4-5
				\item 20 minute group recording
			\end{itemize}
		\item \textbf{Assessment 3: Final Examination (60\%)} - A mix of MCQs and case study/short-answer questions on all topics.
			\begin{itemize}
				\item A new section will appear on Moodle in Week 8 to help students prepare
				\item All sample questions will be provided by Week 9
			\end{itemize}
		\item \textbf{Formal Requirements} - Passing mark of at least 50/100 throughout all assessments
	\end{itemize}

\chapter{Accounting}
\section{Introduction to Financial Management}
	\textbf{Learning Objectives}
		\begin{itemize}
			\item Differentiate between accounting, finance, and tax
			\item Understand the different forms of accounting information
			\item Differentiate between revenues and expenses
			\item Explain the concepts of accrual accounting, cash accounting, and accounting profit
		\end{itemize}
	
	\subsection{What is financial management?}
		\begin{itemize}
			\item Financial management is the process of planning, organising, controlling, and monitoring financial resources to achieve company goals and objectives $\rightarrow$ how we use information to make informed business decisions
			\item The \textit{primary objective} of financial management is to maximise shareholder value through appropriate utilisation and decision making
		\end{itemize}

	\subsection{Why is financial management important to study}
		\begin{itemize}
			\item Financial management is a key life skill
			\item Entrepreneurship and roles within a company
			\item Implications on all aspects of business operations (such as Marketing, Management, Tax, Information Systems, Actuary)
			\item Eg. Budgeting within a business
			\item Measuring performance of a business, eg. profit
		\end{itemize}

		\textbf{Accounting is how you understand and run a successful business}

		Signs of business health include
			\begin{itemize}
				\item Profitability
				\item Cash flow
				\item Short and long term sustainability
				\item Business value
			\end{itemize}
		
		Accounting powers every business decision, categorised as long-term (strategic), short-term (operational), commercial (evaluation of business deals). Ultimately, accounting enables career success, as an advisor, entrepreneur, or a leader. Accounting is a good pathway into a career, as a way of entering a company. \\
		
		Finance helps you value a business and make good investment decisions using accounting information.

	\subsection{Accounting vs. Finance}

	\subsection{Accounting and Tax}

	\subsection{Accounting Information}
		\subsubsection{Who uses accounting information?}
			\begin{itemize}
				\item A wide range of stakeholders use accounting information daily
				\item These users often have different needs that accounting information can satisfy
			\end{itemize}

		\subsubsection{What are the different types of information in accounting}
			\begin{itemize}
				\item Financial accounting $\rightarrow$ Information given to people \colorbox{pastel_red}{outside} of the enterprise
				\item Management accounting $\rightarrow$ Information given to users \colorbox{pastel_blue}{within} the enterprise to assist in operational planning and controlling decisions
				\item Audit and internal control $\rightarrow$ An audit confirms the accuracy of financial statements and internal control involves implementing policies and procedures that safeguard assets and ensure reliable financial information
				\item Social and environmental accounting (accounting for sustainability) $\rightarrow$ Provision of non-financial information to users \colorbox{pastel_red}{external} to the enterprise 
			\end{itemize}

		\subsubsection{Financial accounting information and decision-making}
			\begin{itemize}
				\item \textbf{Financial accounting} primarily serves \colorbox{pastel_red}{external stakeholders}, such as investors/shareholders, creditors, regulatory bodies and the general public
				\item It provides an overview of a company's financial health
				\item Analysing financial statements allows managers to evaluate profitability, liquidity, and solvency
				\item Investors rely on financial statements to inform their investment decisions
			\end{itemize}

		\subsubsection{Management accounting information and decision-making}
			\begin{itemize}
				\item \textbf{Management accounting} is designed for \colorbox{pastel_blue}{internal stakeholders}, including managers and executives
				\item It provides detailed financial information to aid in strategic planning, budgeting, performance evaluation, and internal decision-making
				\item Cost accounting, a subset of management accounting, allows managers to optimise pricing and cost efficiency
			\end{itemize}

		\subsubsection{Audit and internal control information and decision-making}
			\begin{itemize}
				\item \textbf{Audits and internal control systems} ensure the reliability and integrity of financial inform, instilling confidence in decision-makers
				\item Audits and internal control systems can also benefit internal stakeholders by safeguarding assets and promoting operational efficiency
				\item Investors trust audited financial statements
			\end{itemize}

		\subsubsection{Social and environmental accounting information and decision-making}
			\begin{itemize}
				\item Social and environmental accounting caters to both internal and external stakeholders
				\item Internally, it helps management assess the company's impact on environmental and societal sustainability
				\item Externally, it informs socially conscious investors, customers, and regulatory bodies about the company's commitment to social and environmental responsibility
			\end{itemize}


	\subsection{Revenue and Expenses}

		\subsubsection{Revenue}
			\begin{itemize}
				\item Revenue represents an increase in company wealth
				\item Wealth increases because customers:
					\begin{itemize}
						\item Pay cash for goods or services
						\item Promise to pay cash (accounts receivable)
					\end{itemize}
				\item Types of revenue include:
					\begin{itemize}
						\item Sales revenue $\rightarrow$ Generated by selling goods and/or services in the ordinary course of business
						\item Other revenue may consist of items such as:
							\begin{itemize}
								\item Interest income on bank accounts or investments
								\item Dividends received from investments in other companies
							\end{itemize}
					\end{itemize}
			\end{itemize}

		\subsubsection{Expenses}

			\begin{itemize}
				\item Expenses represent decreases in company wealth
				\item Expenses must be incurred in order to earn revenue
				\item These \textbf{do not} include the payment of dividends
			\end{itemize}

	\subsection{Cash Accounting}
		\begin{itemize}
			\item Cash accounting involves recording revenues and expenses \textbf{at the time the cash is received or paid}
			\item This is reasonably precise given that the accountant knows whether cash has been paid or received, and the amount is easily determined
		\end{itemize}

		\subsubsection{Limitations of cash accounting}
			\begin{itemize}
				\item The complexity of business means that the financial health of a company is affected by transactions and cash flows in both the past and future
				\item Events where the \textbf{timing} of cash flows differs from the \textbf{substance} of the transaction include:
					\begin{itemize}
						\item Sale of goods or services on credit
						\item Services that will be paid in a later period (eg. electricity, water, gas, hotel accommodations)
						\item Companies receiving cash in advance for services provided (eg. magazine subscriptions, airline tickets, concert tickets)
					\end{itemize}
			\end{itemize}

	\subsection{Accrual Accounting}
		\begin{itemize}
			\item To cope with these complexities, most businesses use accrual accounting
			\item Accrual accounting involves recording revenues and expenses \textbf{at the time they occur}, not when the cash is received or paid
			\item The \colorbox{pastel_green}{key test for revenue recognition is whether the goods and services have been \textbf{rendered}} (ie. delivery to a customer or provision of a service)
			\item \textbf{Receipt of cash is not required for revenue to be recognised}
		\end{itemize}

		\begin{table}[H]
			\centering
			\setstretch{1.25}
			\begin{tabular}{p{4cm}|p{6cm}|p{6cm}}
				\hline
				\multicolumn{1}{c|}{} & 
				\multicolumn{1}{c|}{\textbf{Cash basis}} & 
				\multicolumn{1}{c}{\textbf{Accrual basis}} \\
				\hline
				\textbf{Record revenues when:} & Cash is received 	& Revenues are earned \\
				\textbf{Record expenses when:} & Cash is paid		& Expenses are incurred \\
				\textbf{In Australia, the ATO says that:}	& Businesses with a turnover of less than \$10 million can use cash accounting	& Businesses with a turnover of more than \$10 million must use accrual accounting \\
				\textbf{Benefits:}	& Simple and effective in managing the cash flow position of a business	& More accurate and clear view of the financial health of the business \\

				\hline
			\end{tabular}
		\end{table}
		
\end{document}


